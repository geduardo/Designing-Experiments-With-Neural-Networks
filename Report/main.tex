%%%%%%%%%%%%%%%%%%%%%%%%%%%%%%%%%%%%%%%%%%%%%%%%%%%%%%%%%%%%%%%%%%
%%%%%%%%%%%%%%%%%%%%%%%%%%%%%%%%%%%%%%%%%%%%%%%%%%%%%%%%%%%%%%%%%%
%% FRAME-FILE
%%
%% Author        : Stefan M. Moser,
%%
%% Written       : January 1998, based on a file by Michael Semling
%%
%% Modifications : ongoing, used for semester project 1, semester
%%                 project 2, master thesis, PPS, SADA @ ISI
%%
%% Last Modify.   : 21 May 2004, 24 Oct 2006 (Michèle Wigger)
%%
%% Version       : 3.4
%%
%% Copyright     : Free for distribution if 
%%                     - you keep this header (you may add to it!)
%%                     - you do not charge any fee
%%
%%%%%%%%%%%%%%%%%%%%%%%%%%%%%%%%%%%%%%%%%%%%%%%%%%%%%%%%%%%%%%%%%%
%%%%%%%%%%%%%%%%%%%%%%%%%%%%%%%%%%%%%%%%%%%%%%%%%%%%%%%%%%%%%%%%%%
%% HELP:
%%
%% Compilation of file:
%%
%%  latex frame; bibtex frame; latex frame; latex frame
%%
%%  dvips -Pcmz -Pamz -Psuper -o frame.ps frame.dvi
%% 
%%
%%%%%%%%%%%%%%%%%%%%%%%%%%%%%%%%%%%%%%%%%%%%%%%%%%%%%%%%%%%%%%%%%%
%%%%%%%%%%%%%%%%%%%%%%%%%%%%%%%%%%%%%%%%%%%%%%%%%%%%%%%%%%%%%%%%%%
%% HELP ONLINE:
%%
%% http://computing.ee.ethz.ch/.soft/latex/tetex/     (tardis)
%%
%% file:/usr/pack/tetex-1.0.7-mo/texmf/doc/index.html (ISI)
%%
%% CTAN for any package: http://www.ucc.ie/cgi-bin/ctan
%%
%%%%%%%%%%%%%%%%%%%%%%%%%%%%%%%%%%%%%%%%%%%%%%%%%%%%%%%%%%%%%%%%%%
%%%%%%%%%%%%%%%%%%%%%%%%%%%%%%%%%%%%%%%%%%%%%%%%%%%%%%%%%%%%%%%%%%
%% Styles: article, book, report
\documentclass[11pt,a4paper,twoside]{report}
\pdfoutput=1
%
%% Draft:
%
%\documentclass[draft,11pt,a4paper,twoside,dvips]{report}
                                    % On border a black ruler shows
                                    % lines with problems
%\usepackage[notref, notcite]{showkeys}
                                    % All labels are printed on the
                                    % side 
%
%%%%%%%%%%%%%%%%%%%%%%%%%%%%%%%%%%%%%%%%%%%%%%%%%%%%%%%%%%%%%%%%%%
%% PACKAGES                         DESCRIPTION
%%%%%%%%%%%%%%%%%%%%%%%%%%%%%%%%%%%%%%%%%%%%%%%%%%%%%%%%%%%%%%%%%%
\usepackage[ngerman,USenglish]{babel} % Different languages:
                                    % Swap between the languages
                                    % with \selectlanguage{sprache} 
%\usepackage[latin1]{inputenc}      % acceptance of german Umlaute


%\usepackage{bibtex/macros}  %for IEEE transaction submission it is not allowed to have a separate style-file :-( :-( :-(

\usepackage{etex}				%mehr register etc.

\usepackage{times}
%\usepackage{psfrag}
\usepackage{graphicx}
\usepackage{graphics}
%\usepackage{pstricks,pst-node,pst-tree}
\usepackage{epsfig}
\usepackage{amssymb}
\usepackage{amsmath}
\usepackage{amsthm}
\usepackage{color}
\usepackage{fancyhdr}
%\PassOptionsToPackage{hyphens}{url}

%Raban: Added Joe's style
\usepackage[bookmarks,colorlinks,breaklinks]{hyperref}  % PDF hyperlinks, with coloured links
\definecolor{dullmagenta}{rgb}{0.4,0,0.4}   % #660066
\definecolor{darkblue}{rgb}{0,0,0.4}
\hypersetup{linkcolor=red,citecolor=blue,filecolor=dullmagenta,urlcolor=darkblue} % coloured links



%\usepackage[colorlinks,urlcolor=blue] {hyperref}    %change here colorlinks=false to remove the colored links (for print-version)
\usepackage{url}
%\usepackage[colorlinks, hyperindex,pagebackref] {hyperref}
%\usepackage{backref}
\urlstyle{tt}


%zusaetzlich nicht standard
%\usepackage{amsfonts}		%mathematisches Menge R, N
%\usepackage{mathrsfs}
%\usepackage[usenames]{color}
%\usepackage{textcomp}
%\usepackage{latexsym}		%f�r Re(z) und Img(z)
%\usepackage{amsthm}
%\usepackage{wrapfig}
%\usepackage{pdfpages}
%\usepackage{caption}
%\usepackage{subfig}

%fuer TikZ zum Zeichnen von Graphiken
\usepackage{tikz}
\usetikzlibrary{chains}
\usetikzlibrary{fit}
\usepackage{pgflibraryarrows}		%optional
\usepackage{pgflibrarysnakes}		%optional
\usepackage{epsfig}

% Added by Raban
\usepackage{enumerate}  
\usepackage{nicefrac} 
\usepackage{braket}


\usepackage{bm}  % Define \bm{} to use bold math fonts
\usepackage{bbm}
\usepackage{xcolor}
\usepackage{verbatim}
\usepackage{natbib}
\makeatletter
\makeatother

\begin{comment}
%-------------- start insert modified commands ------------------
\makeatletter
\def\blx@bblfile@bibtex{%
 \blx@secinit
 \begingroup
 \blx@bblstart
%%%%%%%%%%%%%%%%%%%%%%%%%%%%%%%%%%%%%
%
\input{biblio_MA.bbl}
%
%%%%%%%%%%%%%%%%%%%%%%%%%%%%%%%%%%%%%
 \blx@bblend
 \endgroup
 \csnumgdef{blx@labelnumber@\the\c@refsection}{0}}
\makeatother
%-------------- end insert modified commands ------------------
\end{comment}


%\usetikzlibrary{external}
%\usetikzlibrary{decorations.pathreplacing}
%\tikzexternalize[prefix=ConfPaper-]


%\include{psfig}

\usepackage{listings}
\usepackage{url}



\usepackage[latin1]{inputenc}

\usepackage{paralist} % for enumerate

\usepackage{mathabx} % for nice math symbols
%\usepackage{MnSymbol} % for nice math symbols

\usepackage{footmisc} % for \footref



%%%%%%%%%%%%%%%%%%%%%%%%%%%%%%%%%%%%%%%%%%%%%%%%%%%%%%%%%%%%%%%%%%
%% Macros:
%%%%%%%%%%%%%%%%%%%%%%%%%%%%%%%%%%%%%%%%%%%%%%%%%%%%%%%%%%%%%%%%%%
%% for IEEE transaction submission it is not allowed to have a separate style-file. Therefore we have to copy the commands we used
\def\Hb{\ensuremath{H_{\rm b}}}
\def\S{\mathsf{S}}
\def\D{\mathsf{D}}
\def\C{\mathsf{C}}
\def\T{\mathsf{T}}
\def\E{\mathsf{E}}
\def\F{\mathsf{F}}
\def\R{\mathsf{R}}
\def\Z{\mathsf{Z}}
\DeclareMathOperator{\dec}{dec}
\DeclareMathOperator{\decp}{dec'}
\def\Rset{\mathcal{R}_\epsilon}
\def\Dset{\mathcal{D}_\epsilon}
\def\Bparam{Bhattacharyya parameter}

\newcommand{\supp}{\textnormal{supp}}
 \newcommand{\Mat}{\textnormal{Mat}}
 \newcommand{\s}{\textnormal{s}}
  \newcommand{\diag}{ \textnormal{diag}}
\newcommand{\W}{\mathsf{W}} %for channel law.
\newcommand{\Hh}[1]{H\!\left({#1}\right)} %Entropy
\newcommand{\Hcond}[2]{H\!\left({#1}|{#2}\right)} %cond Entropy
\newcommand{\Prvcond}[2]{\,{\rm Pr}\!\left[\left.#1\,\right|\,#2\right]} %cond. prob.
\newcommand{\Prv}[1]{\,{\rm Pr}\!\left[#1\right]} %usage: Pr[X \leqslant 5]
\newcommand{\SNR}{\textnormal{SNR}}
\newcommand{\Bernoulli}[1]{\textnormal{Bernoulli}\left(#1\right)}  %Bernoulli dist.
\newcommand{\pos}[2]{\textnormal{pos}_{{#1}}\left({#2}\right)} %pos function

% Comments
\newcommand{\RI}[1]{{\color[RGB]{0,128,0} *~*~*~ #1 *~*~*~}}

% QRM notaion
\newcommand{\ketbra}[1]{|#1\rangle\langle #1|}
\newcommand{\cnot}{{\footnotesize \textnormal{CNOT}} }
\newcommand{\RM}{\mathcal{RM} }
\newcommand{\QRM}{\mathcal{QRM} }

\newcommand*{\ee}{\mathrm{e}}


%\DeclareMathOperator{\T}{T}
\DeclareMathOperator{\enc}{enc}
\DeclareMathOperator{\EC}{EC}
\def\A{\mathsf{A}}
\def\H{H}
\def\P{P}
\newcommand{\+}{\textnormal{+} }
\newcommand{\bp}{\textbf{+} }
\newcommand{\g}{\hspace{1mm}}
\renewcommand{\d}{\textnormal{d}} %for integration measure
\newcommand{\id}{\textnormal{id}}
\def \I{\mathrm{i}}

% Joe notaion
\def\tr{{\rm tr}}
\def\pr{{\rm Pr}}
\def\Re{{\rm Re}}
\def\cl{{\text{cl}}}
\def\ker{{\text{ker}}}
\def\fpg{F_{{\rm pg}}}
\def\cC{\mathcal C}
\def\cD{\mathcal D}
\def\cE{\mathcal E}
\def\spec{\text{spec}}


\newcommand{\norm}[1]{\left\lVert#1\right\rVert}
\newcommand{\normT}[1]{\left\vert\kern-0.25ex\left\vert\kern-0.25ex\left\vert #1 
    \right\vert\kern-0.25ex\right\vert\kern-0.25ex\right\vert}

% correct bad hyphenation here
\hyphenation{op-tical net-works semi-conduc-tor Spe-ci-fi-cally di-mensio-nal}
\hyphenation{va-rie-ty}
\hyphenation{the-sis}
\hyphenation{opti-ma-li-ty}
\hyphenation{dif-fe-rent}
\hyphenation{ma-the-ma-ti-cal}
\hyphenation{pa-ra-me-ter}
\hyphenation{equali-ty}
\hyphenation{in-tro-duc-tion}
\hyphenation{anal-y-sis}
\hyphenation{in-equa-li-ty}
\hyphenation{using}
\hyphenation{ge-ne-ra-lized}
\hyphenation{ope-ra-tors}
\hyphenation{the-ory}
\hyphenation{pro-ba-bi-li-ty}
\hyphenation{in-te-re-stin-gly}
\hyphenation{multi-variate}
\hyphenation{exact}

%%%%%%%%%%%%%%%%%%%%%%%%%%%%%%%%%%%%%%%%%%%%%%%%%%%%%%%%%%%%%%%%%%
%% SYNTAX PACKAGE ``theorem'':
%%%%%%%%%%%%%%%%%%%%%%%%%%%%%%%%%%%%%%%%%%%%%%%%%%%%%%%%%%%%%%%%%%
%% Example:
%%
%% \newtheorem{name}{titel}
%% \newtheorem{name}{titel}[counter] 
%% \newtheorem{name}[othername]{titel}
%%
%% The first version defines a new type of "theorem" that is invoked by
%%    \begin{name}
%%      ...
%%    \end{name}
%% and that will be called "titel" in boldface. The numeration starts
%% from 1 and goes up through the whole document.
%% The second version does the same, but the numeration is restarted at
%% each counter (where counter=chapter, section, subsection etc.),
%% i.e., for example 1.1, 1.2, 1.3, 2.1, 2.2, 4.1 according to the
%% chapters.
%% The third version puts the numeration to the same counter as the one
%% that has been defined for the "theorem" othername,
%% e.g., Lemma 1.1, Definition 1.2, Definition 1.3, Lemma 1.4 etc.
\newtheorem{mythm}{Theorem}[section]
\newtheorem{myprop}[mythm]{Proposition}
\newtheorem{mycor}[mythm]{Corollary}
\newtheorem{mylem}[mythm]{Lemma}
\newtheorem{myclaim}[mythm]{Claim}
\newtheorem{mysubclaim}[mythm]{Subclaim}
\newtheorem{myfact}[mythm]{Fact}
\newtheorem{myconj}[mythm]{Conjecture}

\theoremstyle{definition}
\newtheorem{mydef}[mythm]{Definition}
\newtheorem{myex}[mythm]{Example}
\newtheorem{myrmk}[mythm]{Remark}


% %
%%%%%%%%%%%%%%%%%%%%%%%%%%%%%%%%%%%%%%%%%%%%%%%%%%%%%%%%%%%%%%%%%%
%% SYNTAX HEADER AND FOOTER
%%%%%%%%%%%%%%%%%%%%%%%%%%%%%%%%%%%%%%%%%%%%%%%%%%%%%%%%%%%%%%%%%%
%%
%% Attention: These headers and footers are only used if you use
%% \pagestyle{fancyplain} 
%%
%% Header and footer have three parts:
%% \lhead, \chead, \rhead bzw. \lfoot, \cfoot,\rfoot. 
%% \lhead ist responsible for TOP LEFT, \chead for the TOP MIDDLE etc.
%% Synopsis: \lhead[\fancyplain{1}{2}]{\fancyplain{3}{4}}
%% 1: Start of chapter on even page
%% 2. Normal even page
%% 3. Start of chapter on odd page
%% 4. Normal odd page 
%%
%% \leftmark = chapter
%% \rightmark = section with number
%% \thepage = page-number
%% \thechapter = chapter-number
%
%\renewcommand{\chaptermark}[1]{\markboth{#1}{}}
%\renewcommand{\sectionmark}[1]{\markright{\thesection\ #1}}
%
\renewcommand{\headrulewidth}{0.4pt}
\renewcommand{\footrulewidth}{0pt}
%
\lhead[\fancyplain{}
{}]
{\fancyplain{}
  {}}
%
\chead[\fancyplain{}
{}]
{\fancyplain{}
  {}}
%
\rhead[\fancyplain{}
{}]
{\fancyplain{}
  {}}
%
\lfoot[\fancyplain{}
{}]
{\fancyplain{}
  {}}
%
\cfoot[\fancyplain{\thepage}
{\thepage}]
{\fancyplain{\thepage}
  {\thepage}}
%
\rfoot[\fancyplain{}
{}]
{\fancyplain{}
  {}}
%
%%%%%%%%%%%%%%%%%%%%%%%%%%%%%%%%%%%%%%%%%%%%%%%%%%%%%%%%%%%%%%%%%%
%% PARTIAL COMPILATION
%%%%%%%%%%%%%%%%%%%%%%%%%%%%%%%%%%%%%%%%%%%%%%%%%%%%%%%%%%%%%%%%%%
%% While working on a part of the document, compile only this 
%% part: 
%\includeonly{chapter1.tex}
%
%%%%%%%%%%%%%%%%%%%%%%%%%%%%%%%%%%%%%%%%%%%%%%%%%%%%%%%%%%%%%%%%%%
%% COUNTER OF FORMULAS
%%%%%%%%%%%%%%%%%%%%%%%%%%%%%%%%%%%%%%%%%%%%%%%%%%%%%%%%%%%%%%%%%%
%% Counter including section-number (for more information see 
%% amsmath):
\numberwithin{equation}{chapter}
%\numberwithin{equation}{section}
%
%%%%%%%%%%%%%%%%%%%%%%%%%%%%%%%%%%%%%%%%%%%%%%%%%%%%%%%%%%%%%%%%%%
%% FRENCH SPACING
%%%%%%%%%%%%%%%%%%%%%%%%%%%%%%%%%%%%%%%%%%%%%%%%%%%%%%%%%%%%%%%%%%
%% Doesn't put any additional space after interpunction-characters
%% like : , . 
%\frenchspacing
%
%%%%%%%%%%%%%%%%%%%%%%%%%%%%%%%%%%%%%%%%%%%%%%%%%%%%%%%%%%%%%%%%%%
%% INDENT OF PARAGRAPH
%%%%%%%%%%%%%%%%%%%%%%%%%%%%%%%%%%%%%%%%%%%%%%%%%%%%%%%%%%%%%%%%%%
%% After paragraph start sentence with indent:
%\setlength{\parindent}{0cm}
%
%%%%%%%%%%%%%%%%%%%%%%%%%%%%%%%%%%%%%%%%%%%%%%%%%%%%%%%%%%%%%%%%%%
%% SIMPLIFIED TYPESETTING
%%%%%%%%%%%%%%%%%%%%%%%%%%%%%%%%%%%%%%%%%%%%%%%%%%%%%%%%%%%%%%%%%%
%% If you activate the simplified setting, then LaTeX uses less 
%% strict rules for the layout (less 'overful hbox' warnings)
%%
%% Activate:
%\sloppy
%% Deactivate:
\fussy
%
%%%%%%%%%%%%%%%%%%%%%%%%%%%%%%%%%%%%%%%%%%%%%%%%%%%%%%%%%%%%%%%%%%
%% PAGE-MARGINS
%%%%%%%%%%%%%%%%%%%%%%%%%%%%%%%%%%%%%%%%%%%%%%%%%%%%%%%%%%%%%%%%%%
%% Left page more to the left, right page more to the right:
\setlength{\oddsidemargin}{1.7cm}
\setlength{\evensidemargin}{1.7cm}
\addtolength{\textheight}{0.0cm}
\addtolength{\textwidth}{0.0cm}
\addtolength{\topmargin}{-0.0cm}
%
%%%%%%%%%%%%%%%%%%%%%%%%%%%%%%%%%%%%%%%%%%%%%%%%%%%%%%%%%%%%%%%%%%
%% STYLE OF HEADER AND FOOTER
%%%%%%%%%%%%%%%%%%%%%%%%%%%%%%%%%%%%%%%%%%%%%%%%%%%%%%%%%%%%%%%%%%
%% NB: Order of this command and the Page-Margins commands is 
%% important!
%
\pagestyle{fancyplain}
%
%%%%%%%%%%%%%%%%%%%%%%%%%%%%%%%%%%%%%%%%%%%%%%%%%%%%%%%%%%%%%%%%%%
%% SPACE BETWEEN LINES
%%%%%%%%%%%%%%%%%%%%%%%%%%%%%%%%%%%%%%%%%%%%%%%%%%%%%%%%%%%%%%%%%%
%% Ph.D. should have 1.5 times bigger space between lines
%% Vieweg Latex Buch p. 33
\renewcommand{\baselinestretch}{1.2}
\large \normalsize
%
%%%%%%%%%%%%%%%%%%%%%%%%%%%%%%%%%%%%%%%%%%%%%%%%%%%%%%%%%%%%%%%%%%
%% INDEX
%%%%%%%%%%%%%%%%%%%%%%%%%%%%%%%%%%%%%%%%%%%%%%%%%%%%%%%%%%%%%%%%%%
%% Automated index, needs 
%% makeindex frame
%
%\makeindex
%
%%%%%%%%%%%%%%%%%%%%%%%%%%%%%%%%%%%%%%%%%%%%%%%%%%%%%%%%%%%%%%%%%%
%%%%%%%%%%%%%%%%%%%%%%%%%%%%%%%%%%%%%%%%%%%%%%%%%%%%%%%%%%%%%%%%%%
%% START OF DOCUMENT
%%%%%%%%%%%%%%%%%%%%%%%%%%%%%%%%%%%%%%%%%%%%%%%%%%%%%%%%%%%%%%%%%%
%%%%%%%%%%%%%%%%%%%%%%%%%%%%%%%%%%%%%%%%%%%%%%%%%%%%%%%%%%%%%%%%%%
%
\begin{document}

%
%%%%%%%%%%%%%%%%%%%%%%%%%%%%%%%%%%%%%%%%%%%%%%%%%%%%%%%%%%%%%%%%%%
%% LANGUAGE OF DOCUMENT
%%%%%%%%%%%%%%%%%%%%%%%%%%%%%%%%%%%%%%%%%%%%%%%%%%%%%%%%%%%%%%%%%%
%
\selectlanguage{USenglish}
%\selectlanguage{ngerman}
%
%%%%%%%%%%%%%%%%%%%%%%%%%%%%%%%%%%%%%%%%%%%%%%%%%%%%%%%%%%%%%%%%%%
%% PAGENUMBERING PREFACE
%%%%%%%%%%%%%%%%%%%%%%%%%%%%%%%%%%%%%%%%%%%%%%%%%%%%%%%%%%%%%%%%%%
%% Roman pagenumbering (arabic, roman,...)
\pagenumbering{Roman}
%
%%%%%%%%%%%%%%%%%%%%%%%%%%%%%%%%%%%%%%%%%%%%%%%%%%%%%%%%%%%%%%%%%%
%% TITEL
%%%%%%%%%%%%%%%%%%%%%%%%%%%%%%%%%%%%%%%%%%%%%%%%%%%%%%%%%%%%%%%%%%
%% AUTOMATICAL:
%
%\title{Eidgen\"ossiche Technische Hochschule \\ Z\"urich \\ \vspace{20mm}
 %Master's Thesis \\ at the Signal and Information Processing Laboratory \\
 %\vspace{10mm} The Hypothesis of Fixed-Key Equivalence for the Group
% Generalization of Linear Cryptanalysis \vspace{5mm}}
%\author{Stefan M. Moser}
%
%
%\maketitle
%
%% BY HAND:
%
\begin{titlepage}
  \mbox{}

  \vspace{-1.5cm}
  \noindent
  \begin{tabular}{@{} l @{} l @{}}
    \begin{minipage}[c]{0.5\textwidth}
      \hspace{-4mm}
      \includegraphics[height=19mm]{figures/eth_logo.pdf}
    \end{minipage} &
    \begin{minipage}[c]{0.5\textwidth}
       \hfill \large Quantum Information Theory Group
    \end{minipage} \\
  \end{tabular}
  \rule{\textwidth}{0.5pt}
  \begin{center}
    {\Large 
      Spring 2020 \hfill Prof.~Dr.~Renato Renner
    }
    
    \vspace{\stretch{5}}
    \LARGE
    Master's Thesis
 
    \vspace{\stretch{8}}
    \Huge\textbf{
    Designing experiments with neural networks
          }
    
    \vspace{\stretch{10}}
    \LARGE{
      Eduardo Gonzalez Sanchez
    }
    
    \vspace{\stretch{10}}
    \rule{\textwidth}{0.5pt}
   
    \vspace{0.0cm}
    \begin{flushleft}
      \begin{tabular}{ll}
        \Large Advisor: & \Large 
        Raban Iten
      \end{tabular}
    \end{flushleft}
  \end{center}
\end{titlepage}
%
%%%%%%%%%%%%%%%%%%%%%%%%%%%%%%%%%%%%%%%%%%%%%%%%%%%%%%%%%%%%%%%%%%
%% OFFICIAL PROJECT-DESCRIPTION
%%%%%%%%%%%%%%%%%%%%%%%%%%%%%%%%%%%%%%%%%%%%%%%%%%%%%%%%%%%%%%%%%%

% I do not have one ....
\thispagestyle{plain}
\cleardoublepage

%%%%%%%%%%%%%%%%%%%%%%%%%%%%%%%%%%%%%%%%%%%%%%%%%%%%%%%%%%%%%%%%%%
%% ACKNOWLEDGMENTS
%%%%%%%%%%%%%%%%%%%%%%%%%%%%%%%%%%%%%%%%%%%%%%%%%%%%%%%%%%%%%%%%%%
%
%\include{preface}
%
%\renewcommand{\chaptername}{}
\chapter*{Acknowledgments}
%\addcontentsline{toc}{chapter}{\numberline{}Acknowledgments}
%\label{cha:acknowledgments}
Blah blah


\vspace{1.1cm}
\noindent
Zurich,  April 14, 2020

\vspace{2.4cm}
\noindent
Eduardo Gonzalez Sanchez
%\renewcommand{\chaptername}{Chapter}
\thispagestyle{plain}
\clearpage

\thispagestyle{plain}
\cleardoublepage
%
%%%%%%%%%%%%%%%%%%%%%%%%%%%%%%%%%%%%%%%%%%%%%%%%%%%%%%%%%%%%%%%%%%
%% ABSTRACTS
%%%%%%%%%%%%%%%%%%%%%%%%%%%%%%%%%%%%%%%%%%%%%%%%%%%%%%%%%%%%%%%%%%
%% ENGLISCH ABSTRACT:
%
%\include{englishabstract}
\huge
\begin{abstract}
  \setcounter{page}{5}
\thispagestyle{plain}
%  \addcontentsline{toc}{chapter}{\numberline{}Abstract}
  \normalsize
  \vspace{0.5cm}


\noindent In this work we propose a new model that mixes reinforcement learning 
and deep learning to create agents able to design strategies for experiments, 
using as feedback only the quality of the predictions about properties of the 
physical system made exclusively from the data the agents collect. This is could
be summarized as 'agents able to do science'; since science checks its validity 
by making predictions over the physical world. In the thesis we also put the 
first building blocks of a theoretical model for the scientific method in 
machines. This last part is important in the long-term goal of developing a 
variant of quantum theory that can consistently describe agents who are using 
the theory. If the science of the future is done by machines, we must ensure 
consistency in the underlying principles driving automated science.

\vspace{3mm}

\noindent 

\vspace{10mm}
\noindent{\bf{Keywords:}}
Reinforcement learning, deep learning, automated science, feature representation, 
experiment design, artificial intelligence
\end{abstract}



\normalsize


%\renewcommand{\abstractname}{Abstract}
\selectlanguage{USenglish}
%

%
\thispagestyle{plain}
\cleardoublepage
%
%%%%%%%%%%%%%%%%%%%%%%%%%%%%%%%%%%%%%%%%%%%%%%%%%%%%%%%%%%%%%%%%%%
%% HEADER TABLE OF CONTENTS ETC.
%%%%%%%%%%%%%%%%%%%%%%%%%%%%%%%%%%%%%%%%%%%%%%%%%%%%%%%%%%%%%%%%%%
%

%
%%%%%%%%%%%%%%%%%%%%%%%%%%%%%%%%%%%%%%%%%%%%%%%%%%%%%%%%%%%%%%%%%%
%% TABLE OF CONTENTS
%%%%%%%%%%%%%%%%%%%%%%%%%%%%%%%%%%%%%%%%%%%%%%%%%%%%%%%%%%%%%%%%%%
%\addcontentsline{toc}{chapter}{\numberline{}Table of Contents}
\tableofcontents
%\addtocontents{toc}{\protect\enlargethispage{1cm}}
\clearpage
%





\thispagestyle{plain}
\cleardoublepage



\lhead[\fancyplain{\scshape \leftmark}
{\scshape \leftmark}]
{\fancyplain{\scshape Semester Project}
  {\scshape Semester Project}}
%
\rhead[\fancyplain{\scshape Semester Project}
{\scshape Semester Project}]
{\fancyplain{\scshape \leftmark}
  {\scshape \leftmark}}
%
%
%%%%%%%%%%%%%%%%%%%%%%%%%%%%%%%%%%%%%%%%%%%%%%%%%%%%%%%%%%%%%%%%%%
%%%%%%%%%%%%%%%%%%%%%%%%%%%%%%%%%%%%%%%%%%%%%%%%%%%%%%%%%%%%%%%%%%
%% MAIN PART
%%%%%%%%%%%%%%%%%%%%%%%%%%%%%%%%%%%%%%%%%%%%%%%%%%%%%%%%%%%%%%%%%%
%%%%%%%%%%%%%%%%%%%%%%%%%%%%%%%%%%%%%%%%%%%%%%%%%%%%%%%%%%%%%%%%%%
%
%
%%%%%%%%%%%%%%%%%%%%%%%%%%%%%%%%%%%%%%%%%%%%%%%%%%%%%%%%%%%%%%%%%%
%% RESET NUMBERING
%%%%%%%%%%%%%%%%%%%%%%%%%%%%%%%%%%%%%%%%%%%%%%%%%%%%%%%%%%%%%%%%%%
%
\setcounter{chapter}{0}
\setcounter{figure}{0}
%
%%%%%%%%%%%%%%%%%%%%%%%%%%%%%%%%%%%%%%%%%%%%%%%%%%%%%%%%%%%%%%%%%%
%% PAGE LAYOUT MAIN PART
%%%%%%%%%%%%%%%%%%%%%%%%%%%%%%%%%%%%%%%%%%%%%%%%%%%%%%%%%%%%%%%%%%
%
\pagenumbering{arabic}
%
\renewcommand{\thechapter}{\arabic{chapter}}
\renewcommand{\thesection}{\thechapter.\arabic{section}}
\renewcommand{\thefigure}{\thechapter.\arabic{figure}}
%\renewcommand{\thefootnote}{\arabic{footnote}}
%
\renewcommand{\chaptermark}[1]{\markboth{#1}{}}
\renewcommand{\sectionmark}[1]{\markright{\thesection\ #1}}
%
\lhead[\fancyplain{\scshape Chapter \thechapter}
{\scshape Chapter \thechapter}]
{\fancyplain{\textsc{\leftmark}}
  {\rightmark}}
%
\rhead[\fancyplain{\scshape \leftmark}
{\textsc{\leftmark}}]
{\fancyplain{\scshape Chapter \thechapter}
  {\scshape Chapter \thechapter}}
%
%%%%%%%%%%%%%%%%%%%%%%%%%%%%%%%%%%%%%%%%%%%%%%%%%%%%%%%%%%%%%%%%%%
%% MAIN CHAPTERS
%%%%%%%%%%%%%%%%%%%%%%%%%%%%%%%%%%%%%%%%%%%%%%%%%%%%%%%%%%%%%%%%%%
%
\cleardoublepage
%\include{chapters}



\chapter{Preface} 

Many of the limitations of humans at doing science come from their biological 
condition:
    \begin{itemize}
        \item  Humans have a limited lifespan: 72.6 years on average 
        \cite{owidlifeexpectancy}. When experienced scientists die their
        knowledge and expertise die with them.
        \item From the limited lifespan, humans can dedicate only a fraction to
        do science. This is inevitable since humans need to eat, sleep and deal
        with social interactions, among other things. Moreover, humans need
        decades of study and training to start making contributions to the 
        scientific knowledge.
        \item Humans are susceptible to suffer from diseases and other limiting
        biological conditions that hinder their scientific production.
        \item Humans' understanding of the physical world is tightly linked to 
        their limited sensorial perception and other inherited or acquired 
        factors like the language or the cognitive capacity.
    \end{itemize}
    
Other limitations are indirectly caused by the need to satisfy their biological
necessities. For example, a person who wants to dedicate their life to science
needs some type of financial support to satisfy the basic human necessities. 
This support usually comes from a greater institution like a state, a company or
a patron. This financial relationship ties infrangibly science to the economical
structure of the society. Profitable discoveries are encouraged while resources
for unprofitable science are scarce. It can be argued that any form of
scientific research, human or not, will require an investment of energy and
resources in a society in which those are limited. This can be true, but a more
efficient way of doing science will increase science independence from the
economy. 

At the same time, scientific discoveries influence drastically modern society
and its economical structure. They provide new knowledge that allows humanity to
develop new tools and protocols to improve human well being. In the last
centuries, science has changed society by setting the theoretical and
experimental grounds of a technological transformation. It is of public
interest to boost and improve scientific production.

During the last century, the amount of available scientific literature has been 
growing exponentially \cite{sinatra2015century,BornmannRudiger}, with a yearly
growth rate of $ {\raise.17ex\hbox{$\scriptstyle\mathtt{\sim}$}} 9\%$ in the 
last decade. Scholars read on average almost 240 articles per year
\cite{publications7010018}. Some authors \cite{Alkhateeb} suggest that science 
is in the midst of a data crisis. Although the available literature grows
exponentially the cognitive capacity of human beings remains constant. This
forces scientists to derive hypotheses from an exponentially smaller fraction of
the collective knowledge. This will lead to scientists increasingly asking
questions that already been answered and reducing further the efficiency of
scientific production. 

Some areas of science are starting to suffer from a reproducibility crisis 
\cite{Saltelli, begley2012raise} in which scientists are generally 
unable to reproduce their peers' findings. Some voices in the physics 
community \cite{Sabine} point out that foundational physics has been 
stagnated during the last decades. However, some authors defend that there isn't
such a crisis \cite{Fanelli2628}. Nonetheless, it's clear that to sustain an 
exponential growth of reliable scientific production with no exponentially 
increasing human effort is impossible, and the crisis is thus, unavoidable.
 \par
However, the lack of efficiency in scientific production is not the only 
drawback produced by the biological limitations of human beings. Humans' 
intuition and understanding of the physical world is conditioned by the
percepts collected by their sensory system. This limitation becomes evident when
trying to intuitively understand physical systems that show behaviors that differ
from those susceptible to be collected by the sensory system. This is the case 
of, for example, quantum theory. Humankind has developed tools to overcome 
the limitations of the sensorial system to observe new properties of 
physical systems that are out of reach for our biological receptors. For 
instance, using infrared cameras to map infrared signals to a representation 
in the visible spectrum, humans can detect infrared radiation. But these tools 
don't allow to build an intuitive understanding of the phenomena without 
analogies to the phenomena perceived by the sensory system. 
For example, people that are blind from birth have never had any input to their
visual cortex, so they have no visual intuition which limits their ability to 
understand some physical concepts. Similarly, the lack of receptors for
other arbitrary physical properties limits human understanding of the physical 
world and likely hinders scientific advance.
\\ 
\par 
Modern science requires from agents with complex cognitive abilities. So far 
humans are the only known material structure able to perform it. It is true that
some animals perform scientific behavior, like Crows or monkeys solving puzzles 
by trial and error. But those anecdotal examples are far from the formalized 
version of the scientific method employed by humans. However, humans are also 
the living proof of the possibility of agents performing sophisticated science.
There is no reason to think that there's anything special in humans that makes 
them the best possible form of a scientific agent. Rather it is reasonable to 
think that there is plenty of space for improvement, since the human brain was
designed solely by millions of years of random mutations and natural selection.

Recent advances in artificial intelligence, yet far from achieving an 
artificial general intelligence, open the door to an automation of science. In 
the recent years, a vast amount of effort has been dedicated to the development
of machine learning techniques to help scientists of the physical sciences to
process data to create new better models \cite{Carleo_2019}. However, these 
machine learning based techniques are just tools to help human scientists to 
interpret complex data to provide new predictions, and not efforts towards an 
automation of science. Nonetheless, the potential role that artificial intelligence
might play in the process of scientific production has been getting growing 
awareness. In \cite{Melnikov_2018}, the authors use a projective simulation 
model to design complex photonic experiments that produce high-dimensional 
entangled multiphoton states. According the authors, the system autonomously 
discovers experimental techniques which are a standard in modern quantum optical
experiments. In \cite{iten2020discovering} the authors explore the use of 
variational autoencoders to extract autonomously physically relevant parameters
from physical data without prior assumptions about the physical system. In
\cite{nautrup2020operationally} they expand the work to present an architecture
based on communitcating agents that deal with different aspects of a physical 
system and show that it can be combined with reinforcement learning techniques.
More work on similar directions can be found in [citas articulo de Raban II].
\par
However, scientific agents need to be designed carefully to minimize the 
inherited biases and limitations from their human creators. In this thesis we 
present a minimal axiomatic model for science and a new model architecture that
mixes reinforcement learning with deep learning to create agents capable to
design strategies for experiments. Using as feedback only the quality of the 
predictions about properties of the physical system made exclusively from the 
data the agents collect from their sensorial available receptors. 

\vspace{12mm}


\chapter{Minimal model for science} \label{science}

\section{Introduction} \label{sec:intro_science}
In this section, we introduce a minimal set of definitions and assumptions 
to define a scientific method. In the goal of achieving an independent automated
science protocol, we must ensure consistency in the underlying principles to
avoid unwanted biases and preconceptions inherited from humans.
\subsection{First assumption: Dynamicality}
We start with a universe $U$. Since the goal is to create agents
able to decipher the properties of the universe $U$, we must make the minimal
number of assumptions that allow us to set a scientific method. First, we need
the universe to be dynamical. If the universe is static, nothing changes and 
no science is possible. 
This will give use the first assumption:

\begin{itemize}
  \item[\textbf{A1 (Dynamicality)}:] There exists a dynamical universe $U$.
\end{itemize}

We can represent the dynamical nature of the universe by
parametrizing it with a real parameter $\tau \in (-\infty,\infty)$,
so that the state of the universe is a function of $\tau$. Note that we are not 
assuming any property of the universe function $U(\tau)\rightarrow S$, where $S$
is the set of possible states of the universe. For example, it may look that by
setting the parameter $\tau$ unbounded we may be forcing the universe $U$ to 
have unbounded dynamics. However, we could have $U(\tau)$ so that:
\begin{align*}
  U(\tau \leq \tau_{\textnormal{initial}})&=s_\text{initial} \\
  U(\tau \geq \tau_{\textnormal{terminal}})&=s_\text{terminal}
\end{align*}
where $s_\textnormal{initial}, s_\textnormal{terminal} \in S $ are the initial 
and terminal states of the universe $U$. We aren't making any assumptions about
the dynamical bounds on $U$. Also, we aren't making any assumptions on any other
properties of $U(\tau)$ or even on what the elements of $S$ are. We aren't also
making any assumption on the continuity of the dynamics, since we could have:
$$U(\tau_{i} >\tau \geq \tau_{i+1})=s_{\tau_i},\;\;\forall i \in \mathbb{N}$$
where $\{\tau_i\}_{i=0}^\infty$ is an arbitrary monotonically increasing
sequence of real numbers. We aren't assuming as well anything about the 
deterministic nature of $U$, since $U(\tau)$ could be a probabilistic function.
The only assumption made by the statement \textbf{A1} is that the universe 
evolves according some rule $U(\tau)$.

Note that the parameter $\tau$ doesn't necessarily represent the time as 
perceived by humans. It's just a parameter defined to convey the dynamical 
nature of the universe.

\begin{myex}
  \label{example1}
In this example we are going to define a universe that satisfies \textbf{A1}.
The universe $U$ consists of $n^2$ elements $\{a_{jk}\}_{j,k=0}^{n-1}$. Each 
element can exist in one of two substates: \{1, 0\}. The set of states $S$ is 
then $\{0,1\}^{n^2}.$
Now we need to equip
the universe with a dynamical law $U(\tau)$. Assuming a discrete evolution, 
it could be, for example, the laws of a deterministic cellular automata. Or 
a probabilistic law so that each element changes its state with a 
certain probability in each dynamical step. It could be any rule that associates
a state of $S$ with each discrete value $\tau_i$. However, we could also have 
continuous dynamics. For instance, we can set:

\begin{equation}
  P(a_{jk}=1,\tau)=e^{-\tau^2} ,\;\;\forall j,k 
  \label{DynamicalLaw1}
\end{equation}

where $P(a_{jk}=1, \tau)$ is the probability of the element $a_{jk}$ being in 
the state $1$ at the \textit{time} \footnote{For communicative convenience 
we use the word time to design the value of $\tau$. However, it doesn't mean
that $\tau$ represents the time as perceived by humans.} $\tau$. With this 
dynamical law, the evolution of the universe is unbounded, although it has 
terminal and initial states: all elements in the substate 0. One may ask what it 
means that the universe is described by a probability function like 
\eqref{DynamicalLaw1}. It means that at a given value of $\tau$ the state of 
the universe is choosen randomly according to \eqref{DynamicalLaw1}.
\begin{figure}[t]
  \centering
  \includegraphics[width=100mm,scale=0.5]{figures/UniverseA1.pdf}
   \caption{Graphical representation of the dynamical change of a universe
   with $n=8$.}
\end{figure} 
\end{myex}

From the assumption \textbf{A1} we can deduce some consequences.
\begin{myclaim}
Any universe that satisfies \textbf{A1} has at least one element that can exist
in more than one substate.
\end{myclaim}

The proof of this claim is obvious: an empty set cannot change. A set in which 
all elements can exists in only one substate also cannot change. Therefore, 
the simplest universe in which \textbf{A1} holds is a universe with only one
element that can exists in two substates, this is, a bit that changes its value
according to a dynamical law.

\subsection{The scientific method}

Now we need to define what is science in this minimal context. First, let's 
define what an agent is.

\begin{mydef}
  An agent $A$ is defined as a subset of the universe $U$. $A(\tau)$ is the
  composition of $A$ at the dynamical value $\tau$. The dynamical evolution
  of $A$ is determined by the dynamical evolution of $U$.
\end{mydef}

This is a very broad definition of an agent, since we define them just as subsets
of the universe, so anything can be an agent. We can safely make this assumption 
since humans and computers are subsets of the universe. The definition implies 
that agents obey the same physical laws than the universe. Some philosophers 
would call this implication a materialistic assumption.

\par

Another definition required for science is the definition of measurement or 
observation. This definition is particularly delicate in the context of quantum
theory, so we have to be very careful in its definition. 

\begin{mydef}
  An observation $\hat{O}_A$ is defined as the dynamical process in which the 
  state of any subset of an agent $A$ gets correlated to the state of another
  subset of $U$, the object $O$, that may or not be disjoint to $A$.  
\end{mydef}

This definition is also very broad, so let's explain it. When we talk about 
observations in the context of humans, we usually understand them as information
from an object acquired by our sensorial system, for example by observing with
our visual system a bunch of photons emitted from a source of light. However, 
these terms mean nothing in our minimal model so we need to be more specific.
When we see an object, the process, as far as we know, happens because a 
photon coming from the object hits some receptors in our retina producing a 
chain reaction that triggers an specific state in some part of the brain. In 
other words, an specific part of us (the agent $A$) gets correlated to the state
of the object (a subset of $O$ the universe) as a result of the dynamical 
evolution of the universe. The dynamical process that gets both states 
correlated is an observation.

\begin{myex}
  In this example we are going to see how an observation translate to our
  simple model of universe. With the same universe than in \ref{example1}, we 
  can define a new dynamical law that consists on a 1 doing a random walk 
  in a bidimensional grid of zeros. Each time an element $a_{jk}$ switches to 
  1, it gets correlated to the rest of the elements of the universe since 
  all of them must be zero. In this case, the agent $A$ is the subset of $U$ 
  containing only $a_{jk}$ while the object $O$ can be any subset of $U$.
  We say then that $A$ has made an observation $\hat{O}_{A}$.
\end{myex}

We need one more definition to define the scientific method. Science is about
making predictions about the physical world, so we need to define what 
a prediction is in our minimal context.



\begin{mydef}
  A prediction is... \textit{(I haven't come yet with a successful definition
  for a prediction. I'm trying to formulate it with different agents trying 
  communicating correlations about the "future" of some subsystem of U)}
\end{mydef}

\subsection{Second assumption: Emergency}
\textit{I want to have a formal definition of predictions and the scientific
method before writing this subsection. But it's just a corollary of the anthropic
principle: if we are able to do science, then the dynamical law of the universe
must allow for scientific agents to be generated. For example, a simple cellular
automaton that converges to a stable state wouldn't fulfill this assumption.}

\chapter{Machine Learning theory} 

\chapter{Experimenter-Analyzer model}
In this section, we are going to present our proposal for a basic model for a
machine learning set up that designs an experiment applying the scientific 
method. It is a model that mixes reinforcement learning and and deep learning to
create agents apable to design strategies for experiments, using as feedback 
only the quality of the predictions about properties of the physical system 
made exclusively from the data the agents collect.
% STOP DELETING!
%
\clearpage
%
%
%%%%%%%%%%%%%%%%%%%%%%%%%%%%%%%%%%%%%%%%%%%%%%%%%%%%%%%%%%%%%%%%%%
%%%%%%%%%%%%%%%%%%%%%%%%%%%%%%%%%%%%%%%%%%%%%%%%%%%%%%%%%%%%%%%%%%
%% APPENDIX
%%%%%%%%%%%%%%%%%%%%%%%%%%%%%%%%%%%%%%%%%%%%%%%%%%%%%%%%%%%%%%%%%%
%%%%%%%%%%%%%%%%%%%%%%%%%%%%%%%%%%%%%%%%%%%%%%%%%%%%%%%%%%%%%%%%%%
%
\begin{appendix}
%
%%%%%%%%%%%%%%%%%%%%%%%%%%%%%%%%%%%%%%%%%%%%%%%%%%%%%%%%%%%%%%%%%%
%% HEADER APPENDIX
%%%%%%%%%%%%%%%%%%%%%%%%%%%%%%%%%%%%%%%%%%%%%%%%%%%%%%%%%%%%%%%%%%
%
\lhead[\fancyplain{\scshape Appendix \thechapter}
{\scshape Appendix \thechapter}]
{\fancyplain{\scshape \leftmark}
  {\rightmark}}
%
\rhead[\fancyplain{\scshape \leftmark}
{\scshape \leftmark}]
{\fancyplain{\scshape Appendix \thechapter}
  {\scshape Appendix \thechapter}}
%
%%%%%%%%%%%%%%%%%%%%%%%%%%%%%%%%%%%%%%%%%%%%%%%%%%%%%%%%%%%%%%%%%%
%% CHAPTERS APPENDIX
%%%%%%%%%%%%%%%%%%%%%%%%%%%%%%%%%%%%%%%%%%%%%%%%%%%%%%%%%%%%%%%%%%
%
%\include{proof}

\chapter{Pretty good measurement} \label{app:pgm}

In this appendix, we give some additional information about the pretty good measurement, completing the discussion in the preface. Let us first formalize our goal: \\
Fix a set of density operators $\{\rho_x\}$ on a quantum system $B$ and a discrete probability distribution $P_X$ with finite support. Alice chooses an $x$ with probability  $P_X(x)=:p_x$ and prepares the corresponding state $\rho_x$ on the system $B$. Bob has access to system $B$ and wants to find out which $x$ has been chosen by Alice. We can summarize the information from the point of view of Bob in the following cq state $\rho_{XB}=\sum_{x} p_x \ketbra{x}_X \otimes (\rho_x)_B$. The measurement preformed by Bob can be described by POVM elements $\Lambda:=\{\Lambda_x\}$ on the system $B$. Then, the probability that Bob guesses correctly in the case that Alice has chosen $x$ is given by $\tr \, \Lambda_x \rho_x$, and hence, the unconditioned success probability (using the POVM $\Lambda$) is $p^{\Lambda}_{\text{guess}}(X|B):=\sum_{x} P_X(x) \, \tr \, \Lambda_x \rho_x$. Therefore, our goal is to find the POVM elements $\Lambda_x$ that maximize $p^{\Lambda}_{\text{guess}}(X|B)$. We define
\begin{align}
p_{\text{guess}}(X|B):= \underset{\Lambda_x}{\max} \sum_{x} P_X(x) \, \tr \, \Lambda_x \rho_x \,.
\end{align}
Unfortunately, it turns out that this optimization problem is not easy to solve in general. However, a different approach was taken in~\cite{belavkin_optimal_1975, hausladen_`pretty_1994}. Indeed,  they defined the pretty good POVM elements 
\begin{align}
\Lambda^{\text{pg}}_x:=P_X(x) \, \hat{\rho}^{-\frac{1}{2}} \rho_x \hat{\rho}^{-\frac{1}{2}} \, ,
\end{align}
where we set $ \hat{\rho}:= \sum_{x} P_X(x) \, \rho_x\,$. Then, the pretty good success probability is given by 
\begin{align}
p^{\text{pg}}_{\text{guess}}(X|B):=\sum_{x} P_X(x) \, \tr \, \Lambda^{\text{pg}}_x \rho_x \, .  
\end{align}
It turns out that the choice $\Lambda_x=\Lambda^{\text{pg}}_x$ is indeed pretty good in that $p_{\text{guess}}(X|B)$ is bounded from below and above in terms of $p^{\text{pg}}_{\text{guess}}(X|B)$ (cf.~\eqref{eq:pgm_is_pg} for the exact statement). These bounds follow elegantly in the framework of this thesis as discussed in detail in Chapter~\ref{cha:pgm}.

%=\tr \, \Lambda_{XB} \rho_{XB}$, where we defined $\Lambda_{XB}:= \sum_{x\in S} \ketbra{x}_X \otimes \left(\Lambda_x\right)_B\,$

\chapter{Notation and abbreviations} \label{APPnoation}

For an overview of the notation for quantum R\'enyi divergences and quantum conditional R\'enyi entropies used in this thesis, see Section~\ref{sec:important_divergences} and  Section~\ref{sec:cond_entropies}, respectively. Note also that our notation follows the one of~\cite{tomamichel_quantum_2016}. \\


We use the terms "non-negative operators" and "positive operators" to refer to linear, non-negative or positive operators on a Hilbert space, respectively. For simplicity, we consider only finite dimensional Hilbert spaces throughout this thesis. Therefore, non-negative operators and positive operators can always be viewed as positive semi-definite and positive definite matrices (over the complex numbers), respectively. \\
Throughout this thesis, taking the inverse of a non-negative operator $\rho$ should be viewed as taking the inverse evaluated only on the support of $\rho$.\\


Note also that we do not use a specific basis for the logarithm in this thesis. However, the exponential function should be considered as the reverse function of the chosen logarithm.\\

\vspace{5mm}

\noindent A list of abbreviations we use is available at Table~\ref{TABabbrev} and a comprehensive list of symbols can be found in Table~\ref{TABsymbols}. Note that the notation for matrices is also used for operators on Hilbert spaces in this thesis. This causes no confusion, because we work with finite dimensional Hilbert spaces only. 

\begin{table}[!ht]
\renewcommand{\arraystretch}{1.3}
\caption{List of abbreviations}
\label{TABabbrev}
\centering
\begin{tabular}{l l}
\hline
CPTP & Completely positive, trace-preserving (linear map)\\
POVM & Positive operator valued measure \\
DPI & Data-processing inequality [cf.~\eqref{eq:DPI}]\\
ALT & Araki-Lieb-Thirring (inequality) [cf. Theorem~\ref{thm:ALT}]\\
GT & Golden-Thompson (inequality) [cf. Theorem~\ref{thm:GT}]\\
cq & classical quantum \\
\hline
\end{tabular}
\end{table}

\begin{table}[!ht]
\renewcommand{\arraystretch}{1.3}
\caption{Notational conventions for mathematical expressions}
\label{TABsymbols}
\centering
\begin{tabular}{l l}
\hline
Operators on Hilbert spaces &\\ \hline
$\rho$, $\sigma$ & Typical elements of the set of non-negative operators\\
$\ker(\rho)$& Kernel of a non-negative operator $\rho$ \\
$\sigma \gg \rho$& $\ker( \sigma) \subseteq \ker(\rho)$ \\
$\cD(A)$& Set of density operators on a quantum system A, \\
&\quad i.e., non-negative operators $\rho$ with $\tr\rho=1$ \\
$\rho_A$& Density operator on a quantum sytem $A$ \\
$|A|$& Dimension of the Hilbert space $A$ \\ \hline
Matrices&\\ \hline
$\textnormal{Mat}(m,n)$ & Complex $m \times n$ matrices\\
$\textnormal{U}(n)$ & Unitary $n \times n$ matrices\\
$A^{*}$ & Conjugate transpose of a matrix $A \in \textnormal{Mat}(n,n)$ \\
$A\geqslant 0$&The matrix $A$ is positive semi-definite \\
$A> 0$&The matrix $A$ is positive definite \\
$A \#_{\alpha} B $&$= A^{\frac{1}{2}} \left(A^{-\frac{1}{2}}BA^{-\frac{1}{2}} \right)^{\alpha} A^{\frac{1}{2}}$ \quad (for $A,B>0$)\\
& \quad [$\alpha$-\emph{weighted geometric mean}]\\
$[A,B] $&$=AB-BA$ \quad [\emph{Commutator}]\\ \hline
Norms&\\ \hline
$|A|$&$= \sqrt{AA^{*}}$  for any $A \in \textnormal{Mat}(n,n)$\\
$\norm{\cdot}_p$&Schatten $p$-quasi-norm (cf. Section~\ref{sec:Schatten}) \\
$\normT{\cdot}$&Any unitarily invariant norm (cf. Definition~\ref{def:unitarily_inv_norm}) \\
\hline
\end{tabular}
\end{table}









%
%
%\include{derivations}
% \chapter{Another Chapter in the Appendix}
% \label{cha:anotherchapterintheappendix}
% 
% Each chapter of the appendix is better included by the command 
% \emph{$\backslash$include\{file\}}.
% %



\clearpage
%
%%%%%%%%%%%%%%%%%%%%%%%%%%%%%%%%%%%%%%%%%%%%%%%%%%%%%%%%%%%%%%%%%%
%% HEADER BIBLIOGRAPHY
%%%%%%%%%%%%%%%%%%%%%%%%%%%%%%%%%%%%%%%%%%%%%%%%%%%%%%%%%%%%%%%%%%
%
\lhead[\fancyplain{\scshape Appendix}
{\scshape Appendix}]
{\fancyplain{\scshape \leftmark}
  {\scshape \leftmark}}
%
\rhead[\fancyplain{\scshape \leftmark}
{\scshape \leftmark}]
{\fancyplain{\scshape Appendix}
  {\scshape Appendix}}
  
  
  
  
  
  
  
  %%%%%%%%%%%%%%%%%%%%%%%%%%%%%%%%%%%%%%%%%%%%%%%%%%%%%%%%%%%%%%%%%%
%% LIST OF FIGURES
%%%%%%%%%%%%%%%%%%%%%%%%%%%%%%%%%%%%%%%%%%%%%%%%%%%%%%%%%%%%%%%%%%
% \addcontentsline{toc}{chapter}{\numberline{}List of Figures}
% \listoffigures
%\clearpage
%
%%%%%%%%%%%%%%%%%%%%%%%%%%%%%%%%%%%%%%%%%%%%%%%%%%%%%%%%%%%%%%%%%%
%% LIST OF TABLES
%%%%%%%%%%%%%%%%%%%%%%%%%%%%%%%%%%%%%%%%%%%%%%%%%%%%%%%%%%%%%%%%%%
% \addcontentsline{toc}{chapter}{\numberline{}List of Tables}
% \listoftables
%\clearpage
%


%
%%%%%%%%%%%%%%%%%%%%%%%%%%%%%%%%%%%%%%%%%%%%%%%%%%%%%%%%%%%%%%%%%%
%% BIBLIOGRAPHY
%%%%%%%%%%%%%%%%%%%%%%%%%%%%%%%%%%%%%%%%%%%%%%%%%%%%%%%%%%%%%%%%%%
\bibliographystyle{plain}
\bibliography{references}
%%%%%%%%%%%%%%%%%%%%%%%%%%%%%%%%%%%%%%%%%%%%%%%%%%%%%%%%%%%%%%%%%%
%% END OF APPENDIX
%%%%%%%%%%%%%%%%%%%%%%%%%%%%%%%%%%%%%%%%%%%%%%%%%%%%%%%%%%%%%%%%%%
%
\end{appendix}
%
%%%%%%%%%%%%%%%%%%%%%%%%%%%%%%%%%%%%%%%%%%%%%%%%%%%%%%%%%%%%%%%%%%
%% INDEX
%%%%%%%%%%%%%%%%%%%%%%%%%%%%%%%%%%%%%%%%%%%%%%%%%%%%%%%%%%%%%%%%%%
%
%\printindex
%
%%%%%%%%%%%%%%%%%%%%%%%%%%%%%%%%%%%%%%%%%%%%%%%%%%%%%%%%%%%%%%%%%%
%% END FRAME
%%%%%%%%%%%%%%%%%%%%%%%%%%%%%%%%%%%%%%%%%%%%%%%%%%%%%%%%%%%%%%%%%%
%
\end{document}
%
%%%%%%%%%%%%%%%%%%%%%%%%%%%%%%%%%%%%%%%%%%%%%%%%%%%%%%%%%%%%%%%%%%
%%%%%%%%%%%%%%%%%%%%%%%%%%%%%%%%%%%%%%%%%%%%%%%%%%%%%%%%%%%%%%%%%%
%%%%%%%%%%%%%%%%%%%%%%%%%%%%%%%%%%%%%%%%%%%%%%%%%%%%%%%%%%%%%%%%%%
